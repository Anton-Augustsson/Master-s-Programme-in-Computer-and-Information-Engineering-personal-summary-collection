\chapter{Algebra 1}

\newpage

\begin{multicols}{2}
\section{logik}
\begin{itemize}
\item Utsaga: ett påstående som erhåller antigen värderna sann(S) eller falsk(F) vilket ge en stluten utsaga
eller ej ett sannings värde vilket kallas för öpna utsagor
\item Kombinerade utsagor:
\item Kunjuktion utsagor: består av två utsagor som vi kallar $A$ och $B$
\item and: $\land  (A \land B)$
\item Dissjuntion utsagor: består av två utsagor som vi kallar A eller B
\item or: $\lor  (A \lor B)$
\item icke utsaga A (motsatsen): $\neg A$
\item Alla: $\forall$
\item Minst en:  $\exists$
\item $\neg(\forall x : A) \Leftrightarrow \exists x : \neg A$
\item $\neg(\exists x : A) \Leftrightarrow \forall x : \neg A$
\end{itemize}

\textbf{Exempel:}
\begin{align*}
  &\text{Alla reala tal x gäller att } (x+1)^2 = 0 \\
  &\forall x : (x+1)^2 = 0 \\
  &\neg(\forall x : (x+1)^2 = 0) = \exists x : (x+1)^2 \not = 0 \\
  &\neg A \text{: Det finns reala tal x gäller att } (x+1)^2 \not 0 \\
\end{align*}


\subsection{värde tabeller}
\textbf{Kunjuktions värdetabel:}\par
\begin{center}
\begin{tabular}{ |c|c|c| } 
 \hline
 A  & B  & \(A \land B\) \\ 
 S  & S  & S          \\ 
 S  & F  & F          \\  
 F  & S  & F          \\ 
 F  & F  & F          \\ 
 \hline
\end{tabular}
\end{center}

\textbf{Dissjuktions värdetabel:}\par
\begin{center}
\begin{tabular}{ |c|c|c| } 
 \hline
 A  & B  & \(A \lor B\) \\ 
 S  & S  & S          \\ 
 S  & F  & S          \\  
 F  & S  & S          \\ 
 F  & F  & F          \\ 
 \hline
\end{tabular}
\end{center}



\subsection{Implikationer}
\begin{align*}
  &\quad \text{A medför B: } A \Rightarrow B \text{ (implikaiton)} \\
  &\quad \text{A och B medför varandra: } A \Leftrightarrow B \text{ (ekvivalens)} \\
\end{align*}


\textbf{Implication värdetabell:}\par
\begin{center}
\begin{tabular}{ |c|c|c| } 
 \hline
 A  & B  & \(A \Rightarrow B\) \\ 
 S  & S  & S          \\ 
 S  & F  & F          \\  
 F  & S  & S          \\ 
 F  & F  & S          \\ 
 \hline
\end{tabular}
\end{center}

\textbf{Ekvivalens värdetabel:}\par
\begin{center}
\begin{tabular}{ |c|c|c| } 
 \hline
 A  & B  & \(A \Leftrightarrow B\) \\ 
 S  & S  & S          \\ 
 S  & F  & F          \\  
 F  & S  & F          \\ 
 F  & F  & S          \\ 
 \hline
\end{tabular}
\end{center}


\textbf{Exempel1:}
\begin{equation}
  x = \sqrt{6 - x}
\end{equation}

\begin{align*}
  &x = \sqrt{6 - x} \Rightarrow x^2 = 6 - x \Leftrightarrow x^2 + x - 6 = 0 \\
  &\text{pq-formeln: } x = -\frac{1}{2} \pm \sqrt{\frac{1}{4} + \frac{24}{4}} \\
  &\Leftrightarrow (x=-3) \lor (x=2) \\
  &\text{Eftersom det är en implikation år höger och inte} \\
  &\text{ekvivalens så behöver inte rötterna vara sanna } \\
  &\\
  &\text{Testar för falska rötter: } \\
  &2 = \sqrt{6-2} \text{ Sann} \\
  &2 = \sqrt{6-(-3)} \text{ Falsk } 2 \ne \sqrt{6-(-3)} \\
\end{align*}


\textbf{Exempel2:}\par
\begin{equation}
  (x+2)(x+1) = 2x(x+1)
\end{equation}

\begin{align*}
  &(x+2)(x+1) = 2x(x+1) \text{ is not } \\
  &\Rightarrow (x+2) = 2x \text{ (x=0 is not allowed)} \\
  &\text{Insted do ass following: } \\
  &(x+2)(x+1) = 2x(x+1) \\
  &\Leftrightarrow (x+2)(x+1) - 2x(x+1) = 0 \\
  &\Leftrightarrow (x+1)(x+2-2x) = 0 \\
  &\Leftrightarrow (x+1=0)\lor(2-x=0) \\
  &\text{svar ej: x=1 och x=2} \\
  &\text{svar: x=1 eller x=2} \\
  &\text{svar: } x=1 \lor x=2 \\
\end{align*}


\section{Mängder}
\begin{align*}
  &\text{Naturliga tal: } \mathbb{N} = \{0, 1, 2 , 3 ..  \} \\
  &\text{Heltal: }\mathbb{Z} = \{.. -2, 1, 0, 1, 2 ..  \} \\
  &\text{Rationella tal: }\mathbb{Q} = \{ \frac{a}{b} | a,b \in \mathbb{Z}, b \\
  &\text{Irrationella tal: }\mathbb{P} = \frac{\mathbb{R}}{\mathbb{Q}}eller \{ x | x \in \mathbb{R}, x \notin \mathbb{Q} \} \\
  &\text{Reella tal: }\mathbb{R} =  \mathbb{P} \cup \mathbb{Q} \\
\end{align*}

\begin{align*}
  A \cup B &= \{ x:(x \in A) \lor (x \in B)\} \\
  A \cap B &= \{ x:(x \in A) \land (x \in B)\} \\
  A \setminus B &= \{ x:(x \in A) \land (x \notin B)\} \\
  A^{\text{\#}} &= \{ x:(x \in X) \land (x \notin A)\} \\
\end{align*}

\textbf{Exempel:}
\begin{equation*}
  \text{Bevisa: } X \setminus (A \cup B) = (X \setminus A) \land (X \setminus B) 
\end{equation*}

\begin{align*}
  x &\in (X \setminus (A \cup B)) \Rightarrow (x \in X) \land (x \notin (A \cup B)) \\
  &\Rightarrow (x \in X) \land (x \notin A) \land (x \notin B) \\
  &\Rightarrow (x \in X \setminus A) \land (x \in X \setminus B) \\
  &\Rightarrow x \in (X \setminus A) \land (X \setminus B) \\
  &\Rightarrow X \setminus (A \cup B) \subseteq (X \setminus A) \land (X \setminus B)
\end{align*}


\section{Bevis}
\subsection{Induktions bevis}
\begin{itemize}
  \item steg 1: bevisa att det gäller för basfallet.
  \item Steg 2: bevvisar att p => p.
  \begin{itemize}
    \item a. Antar att det stämmer för pp.
    \item b. Vissar att med att stopa in antagandet i pp+1 så blir hl = vl.
  \end{itemize}
\end{itemize}
Tips: Förenkla hl först och sedan vl på klad papper fram och tillbacka. \newline

\textbf{Exempel Recursion:}
\begin{align*}
  &a_1 = 2, a_{n+1} = \frac{7a_n}{7-a_n}, n \in \mathbb{N} \\
  &\text{Vissa med induktion att } a_{n+1} = \frac{14}{7-2n} \\ 
  &\text{Bevis med induktions} \\
  &\text{steg 1: visar att påståendet som vi kallar p } \\
  &\text{gäller för basfallet (n=1)} \\
  &VL_1: a_{1+1}=a_2=\frac{7 \cdot 2}{7-2}=\frac{14}{5}, \\
  &HL_1: a_{1+1}=a_2=\frac{14}{7-2}=\frac{14}{5} \\
  &\\
  &\text{steg 2: visar att } p_m \Rightarrow p_{m+1} \\
  &\text{steg 2a: antar att } p_m \text{ gäller} \\
  &a_{m+1}=\frac{14}{7-2m}\\
  &\text{steg 2b: bevisar attt } p_m \Rightarrow p_{m+1} \\
  &\text{ genom att andvända antagandet} \\
  &VL_{m+1}a_{m+2} 
  = \frac{7a_{m+1}}{7-a_{m+1}} 
  = \frac{7 \frac{14}{7-2m}}{7-\frac{14}{7-2m}} \\
  &= \frac{\frac{7 \cdot14}{7-2m}}{\frac{7(7-2m) - 14}{7-2m}} 
  = \frac{7 \cdot 14}{7(7-2m+2)}
  = \frac{14}{7-2(m+1)} \\
  &HL_{m+1}: \frac{14}{7-2(m+1)} \\
  &\text{Enlight induktionsprincipen är $ p_m $ sann för alla } \\
  &n = 1,2,3 \ldots \text{ VSB}  \\
\end{align*}

\subsection{Motsägelse bevis}
\begin{itemize}
  \item Steg 1: Formulera utsagan och icke utsagan
  \item Steg 2: Hitta en motsägelse med utsagan
\end{itemize}
Tips: Förenka båda led, tänk på teorin vi har, bättre att gå vaga moteveringar en inga alls \newline

\begin{align*}
  &\text{Bevis med motsägelse } \\
  &\text{Antar att motsatsen är sann  } \\
\end{align*}
% Contradiction https://www.youtube.com/watch?v=sRDwsfNDXak
% https://www.youtube.com/watch?v=huGWXh4l1M0


\section{Delbarhet}
$a$ är delbar med $b$, altså kvoten ger ingen rest. 
Vi följand: $a \mid b$


\textbf{Divitions algoritmen:}
\begin{align*}
  &a, b \in \mathbb{Z}  \\
  &a  \geq 0 \land b  \geq  0 \\
  &a \mid b \Rightarrow (q \in \mathbb{Z} : q \geq 0) \land (r \in \mathbb{Z} : 0 \leq r \leq a) \\
  &\text{ Sådant att} \\
  &b = q a + r\\
  &q = \text{ kvoten, } r = \text{ resten} \\
\end{align*}


\subsection{Största Gemensama Delaren (SGD)}
\begin{align*}
  &SGD(a,b) \\
  &a = bq + r \\
  &0 \leq r \leq b \\
\end{align*}

\textbf{Euklides algoritm:}
\begin{align*}
  &SGD(a,b) \\
  &a = bq_1 + r_1 \\
  &b = r_1q_2 + r_2 \\
  &r_1 = r_2q_3 + r_3 \\
  &r_2 = r_3q_4 + r_4 \\
  &. \\
  &. \\
  &. \\
  &r_{k-3} = r_{k-2}q_{k-1} + r_{k} \\
  &r_{k-2} = r_{k-1}q_k + 0 \\
  &SGD(a,b) = r_{k} \\
  &\text{Om } r_{k} = 1 \Rightarrow a \in \text{primtal} \lor b \in \text{primtal} \\
\end{align*}

\textbf{Exempel:}
\begin{align*}
  &\text{Förenkla } \frac{114}{96} \\
  &\\
  &SGD(114,96):    \\
  &114 = 1*96 + 18 \\
  &96  = 5*18 + 6  \\
  &18  = 3*6  + 0  \\
  &\frac{114}{6} = 19 \\
  &\frac{96}{6}  = 16 \\
  &\frac{19}{16} \\
\end{align*}


\textbf{Lemma:}
\begin{align*}
  &a, b \in \mathbb{Z} \\
  &x, y \in \mathbb{Z} \\
  &SGD(a,b) = ax + by  \\
\end{align*}


\textbf{Aritmetiska fundamentalsatsen:}
\begin{align*}
  &a \in \mathbb{Z} \land a \geq 2 \Rightarrow  \\
  &\Rightarrow a \text{ kan endast primtalsfaktoriseras på} \\
  &\text{ETT SÄTT} \\
\end{align*}

\textbf{Lemma 2.7:}
\begin{align*}
  &a \geq 2 \land a \notin \text{ Primatal} \Rightarrow \\
  &\Rightarrow q \in \text{ Primatal} \land q \mid a \land \text{ a går att} \\
  &\text{primtals faktoriera}\\
\end{align*}


\subsection{Primtal}

\textbf{Sats:}
För att bestäma om tal a är ett primtal
\begin{align*}
  &a \geq 1 \land a \in \text{ Primatal} \\
  &\text{om ett tall p finns som dellar a gäller} \\
  &\text{följande} \\
  &2 \leq p \leq \sqrt{a} \leq a \\
\end{align*}


\textbf{Exemple:}
\begin{equation*}
  \text{Bestäm om 211 är ett primtal }
\end{equation*}

\begin{align*}
  &\text{Om 211 är ett primtal så finns det inte} \\
  &\text{en äkta delare a} \\
  &2 \leq a \leq \sqrt{211}  \\
  &\sqrt{211} \approx 16 \\
  &a: \{ \cancel{2}, \cancel{3}, \cancel{5}, \cancel{7}, \cancel{11}, \cancel{13} \} \\
  &\text{a kan inte vara en äktadelare} \\
\end{align*}


\textbf{Euklides algoritm:}
Det finns oändligt många primtal

\textbf{Bevis:}
Motsägelsebevis  
\begin{align*}
  &\text{Antar att det finns ändligt många primtal} \\
  &p_1,p_2,P_3,..,p_n \\
  &M = \displaystyle\prod_{k=1}{n} + 1 \Rightarrow M > p_j, j = 1,2,3,..,n \\
  &\text{Vissar att M är ett primtal} \\
  &1 < b < M \land b \mid M \text{ Där b är minsta äkta delaren} \\
  &\text{av} M \\
  &\Rightarrow M = p_1 * b \Rightarrow p_1 \mid 1 \land p \geq 2
\end{align*}
Detta är falskt eftersom båda utsågorna kan inte vara samtidigt.
Eftersom motsatsen inte fungerar resulterar det i att satsen är sann.


\section{Diofantiska ekvationer} 
\textbf{Sats:}
\begin{align*}
  &ax + by = c \land a,b,c \in \mathbb{Z} \and a \ne 0, b \ne b \\
  &\Rightarrow \\
  &ax + by = c \text{ Där } SGD(a,b) = 1 \\
  &\text{Har den anmäla lösningen:} \\
  &x = C x_0 - n b \land y = C y_0 + n a \\
\end{align*}

\textbf{Exempel:}
En lastbil lastas med 12kg packet och 20kg paket. Totalt väger lasten 296, hur många av varge packet?

\begin{align*}
  &ax + by = C \Leftrightarrow 12x + 20y = 296 \\
  &\text{steg1: Testar om } SGD(a,b) \mid c \\
  &SGD(20,12) = 4 \Rightarrow 4 \mid 269 \\
  &\text{steg2: delar SGD med HL och VL} \\
  &\frac{12x - 20y}{4} = \frac{269}{4} \Leftrightarrow 3x - 5y = 74 \\
  &SGD(5,3): \\
  &5 = 1 \cdot 3 + 2 \\
  &3 = 1 \cdot 2 + 1 \\
  &2 = 2 \cdot 1 + 0 \\
  &\text{steg3: Hjälp ekvation för att hitta} x_0, y_0 \\
  &3x_0 - 5y_0 = 1 \\
  &1 = 3 - 1 * 2 \\
  &1 = 3 - (5 - 3) \\
  &1 = 2 \cdot 3 - 1 \cdot 5 \\
  &x_0 = 2, y_0 = -1 \\
  &\text{steg4: almäna lösningen } x = Cx_0 - bn, y = Cy_0 + an \\
  &x = 74 \cdot 2 - 5n = 148 - 5n \\
  &y = 74 \cdot (-1) + 3n = -74 + 3n \\
  &\text{steg5: hittar godtyckliga lösningar} \\
  &\text{Intervallet som n ligger i för x-termen:} \\
  &n = {\cancel{29}, 28, 27, \ldots }, x = 148 - 5 \cdot 28 = 148 - 140 = 8 \\ 
  &\text{Intervallet som n ligger i för y-termen:} \\
  &n = {\cancel{27}, 28, 29, \ldots }, x = -74 + 3 \cdot 28 = 74 - 84 = 10 \\
  &n = 28, x=8, y=10 \\
  &12 \cdot 8 + 20 \cdot 10 = 296 \\
\end{align*}


\section{Talbaser}
\subsection{konvertera från decimal bass till annan bas}
\begin{align*}
  175_8 &= 1 \cdot 8^2 + 7 \cdot 8^1 + 5 \cdot 8^0 \\
\end{align*}

\subsection{konvertera från annan bas till decimal bas}
\begin{align*}
  1609_{10} &= (3 \cdot 8^3 + 1 \cdot 8^2 + 1 \cdot 8^1 + 1 \cdot 8^0) = 3111_8
\end{align*}

eller så kan man andvända euklides algoritm 
\begin{align*}
  &\text{skriv 517 i talbas 3} \\
  &517 = 172 \cdot 3 + 1 \\
  &172 =  57 \cdot 3 + 1 \\
  & 57 =  20 \cdot 3 + 0 \\
  & 20 =   6 \cdot 3 + 2 \\
  &  6 =   2 \cdot 3 + 0 \\
  &  2 =   0 \cdot 3 + 2 \\
  &\text{Svar: } 517_{tio} = 202011_{tre} \\
\end{align*}


\subsection{Andra exempel}
\textbf{Exempel: Skriv $137_{nio}$  i bass tre}
\begin{align*}
  137_{nio} &= 1 \cdot 9^2 + 3 \cdot 9^1 + 7 \cdot 9^0 \\
  &= 1 \cdot 3^4 + 3 \cdot 3^2 + 7 \cdot 3^0 \\
  &= 1 \cdot 3^4 + 1 \cdot 3^3 + 0 \cdot 3^2 + 2 \cdot 3^1 + 1 \cdot 3^0 \\
  &= 11021_{tre}
\end{align*}


\section{Functioner}
\textbf{Typer av funktioner:}
\begin{itemize}
  \item Injektion: alla element $x$ har olika värden $y$ $f: A \to B, \{{\forall x \in A: x_1 \neq x_2, f(x_1) \neq f(x_2)}\}$
  \item Surjektion: mängd $D$ är definitions mängden  $\{g: C \to D, g(x)=y, (\forall y \in D \land \exists x \in C)\}$
  \item Bijektion: Injektion $\land$ Surjektion
\end{itemize}
 
\textbf{Kareskaprodukten:}
\begin{align*}
  &A \times B = \{ (a,b): a \in A \land b \in B \} \\
  &\text{Låt } A = \{ 1,2,3 \} \land B = \{ x,y,z,w \} \\
  &A \times B : \\
  &\{ (1,x), (1,y), (1,z), (1,w) \\
  &(2,x), (2,y), (2,z), (2,w) \\
  &(3,x), (3,y), (3,z), (3,w) \} \\
\end{align*}

\subsection{Inversen}
En funktions invers kan enda
\begin{align*}
  & f: A \to B \land \text{Bijektiv} \Rightarrow f^{-1}(x) \text{ Finns, där}  \\
  &\text{(1) }  x = f^{-1}(y) \Leftrightarrow y = f(x) \\
  &\text{(2) }  D_{f^-1} = V_f \Leftrightarrow D_f = V_{f^-1} \\
  &\text{(3) }  x = f^{-1}(f(x)), x \in D_f = V_{f^-1} \\ 
  &\text{(3) }  y = f^{-1}(f(y)), y \in D_f = V_{f^-1} \\ 
\end{align*}

\subsection{Relatioiner}
\begin{align*}
  &\text{Relation: } xRy \\
  &\text{Reflexiv: } \forall x \in X: xRx \\
  &\text{Symetrisk: } xRy \Rightarrow yRx, x \in X \land y \in X \\
  &\text{Transitiv: } (xRy) \land (yRz) \Rightarrow xRz, \forall x,y,z \in X  \\
  &\text{Ekvivalnsrelation: Reflexiv och Symetrisk Transitiv} \\
\end{align*}


\section{Summor}
\subsection{Aritmetiska summor}
\begin{equation*}
s _ { n } = a _ { 1 } + a _ { 2 } + a _ { 3 } + \ldots + a _ { n } = \frac { n \left( a _ { 1 } + a _ { n } \right) } { 2 }
\end{equation*}


\subsection{Geometriska summor}
Börjar altid med exponenten 0 och gör om summan så att den passar i följade talföljd:
\begin{equation*}
s _ { n } = a + a k + a k ^ { 2 } + \ldots + a k ^ { n - 1 } = \frac { a \left( k ^ { n } - 1 \right) } { k - 1 }
\end{equation*}


\textbf{Exempel:}
\begin{equation*}
\displaystyle\sum _ { k = n } ^ { 2n } (2^{k} - k)
\end{equation*}

\begin{align*}
  &\text{sätter f = 0 = k -n} \\
  &\displaystyle\sum _ { f = 0 } ^ { n } (2^{f+n} - (f+n)) = 2^{n} * \displaystyle\sum _ { f = 0 } ^ { n } (2^{f}) - \displaystyle\sum _ { f = 0 } ^ { n } (f+n) \\
  &\frac{2^{n} (2^{n+1} -1)}{2-1} - \frac{3n(n+1)}{2} = 2^{2n+1} - 2^{n} - \frac{3n(n+1)}{2}\\
\end{align*}


\section{Kongruensräkning}
\textbf{Räkneregler:}
\begin{align*}
  &a+b \pmod{n} \equiv a \pmod{n} + b \pmod{n} \\
  &a \cdot b \pmod{n} \equiv a \pmod{n} \cdot  b(modn) \\
  &a^b \pmod{n} \equiv (a \pmod{n})^b \\
\end{align*}

\textbf{Exempel: Vilket är det minsta positiva rest som kan erhållas vid division av $19^{18}$ med $17$? }\par
\begin{align*}
  19^{18} &\equiv 2^{18} \pmod{17} \equiv 2^4 \cdot 2^4 \cdot 2^4 \cdot 2^4 \cdot 2^2 \pmod{17} \\
  &\equiv {(-1)}^4 \cdot {(-1)}^4 \cdot {(-1)}^4 \cdot {(-1)}^4 \cdot 2^2 \pmod{17} \\
  &\text{svar: resten blir } 4 \\
\end{align*}


\section{Kardinalitet}
Kardinalitet eller ``mäktighet'' är ett sett att räkna med mängders sorlek och alla
oändliga mängder har samma kardinalitet fast det är en delmängd. Naturliga tall har samma
kardinalitet som reala tal trots att naturliga tal är en del mängd av reala talen

Låt $A$ och $B$ vara mängder. Vi sägger att $A$ och $B$ har samma kardinalitet då det finns en bijektion
$\exists f: A \to B, A \sim B$. Vi säger att $A$ står i relation med $B$ omm $A$ och $B$ har samma kardinalitet $ARB$

\subsection{Uppräkneligamängder}
En mängd $X$ sägs vara upräknerlig omm $X$ har samma kardinalitet som $\mathbb{N}$
$\exists g: \mathbb{N} \to X$ där $g$ är bijektiv.
Exempel på upräknerliga mängder är $\mathbb{N},\; \mathbb{Z},\; \mathbb{Q},\; \{ 1,5,78 \}$
Exempel på ej upräknerliga mängder $\mathbb{R}, (0,1)$


\section{Polynom}
\subsection{Polynom division}
\begin{itemize}
  \item Triviala delare: ej heltals kvot med delaren, har en kostant sådant $\lambda \cdot f, \lambda \notin \mathbb{Z}$
  \item Äkta delare: heltals kvot med delaren $f(x), \exists a \in$ polynom: $a \mid f(x)$
  \item Irreducible: om polynomet endast har triviala delare det finns lösningar heltaslösningar $f(x)=0$
  \item Reducible: om polynomet har äkta delare.
  \item Multiplisitet: vilken grad polynomet har.
\end{itemize}


\textbf{Exempel:}
Man vet att ekvationen $z^4 - 2z^3 - 7z^2 + 26z - 20 = 0$ har roten $z = 2 + i$ Lös ekvationen fullständigt.
$z = 2 + i$ Är en lösning är också konjugatet en lösning enligt faktorsatsen $\overline{\rm z}= a - b i$.
$z = 2 \pm i$ Vilket betyder att följande går att factorisera ut polynomet.
$(z -(2 + i))(z -(2 - i)) = z^2 - 4z + 5$

\textbf{Långdivition (liggande stolen):}\par
\polylongdiv[style=A]{x^{4}-2x^{3}-7x^{2}+26x-20}{x^{2}-4x+5}
$z^2 + 2z - 4 = 0$ Är också en läsning som tilslut ger följande.
$z = -1 \pm \sqrt{5} \land z = 2 \pm i$.
Varge n grads polynom har altid n stycken komplexa lösningar.


\subsection{Faktorsatsen}
\begin{align*} 
  &\quad  f(x) = (x - \alpha)p(x) \\
\end{align*}

\textbf{Exempel, Gemensam root hos två polynom:}
\begin{align*}
  p(x)&=x^4-x^3+x^2+2=0, \\
  g(x)&=x^3+4x^2+4x+3=0 \\
  &\text{Har en Gemensam root} \\
  &\text{Eftersom polynomen har en Gemensam root} \\
  &\text{vet vi att det finns en gemensam} (x-\alpha) \\
  &\\
  f(x)&=(x-\alpha)p_1(x) \\  
  g(x)&=(x-\alpha)p_2(x) \\  
\end{align*}
euklides algoritm:
\begin{align*}
  f(x)&=(x-5)g(x)+17(x^2+x+1) \\
  g(x)&=(\frac{1}{17}(x+3))(17(x^2+x+1)) \\
  &+17(x^2+x+1) + 0 \\
  &\\
  h(x)&=x^2+x+1 \\
  f(x)&= (x-5)(x+3)h(x)+17h(x) \\
  &= h(x)((x-5)(x+3)+17) \\
  &= h(x)(x^2-2x+2) \\
  g(x)&= (x+3)h(x) \\ 
  &\\
  f(x)&=0 \Leftrightarrow (h(x)=0 \lor x^2-2x+2=0) \\
  h(x)&=0 \Rightarrow x=-\frac{1}{2} \pm \frac{\sqrt{3}}{2}i \\
  &x^2+2x+2=0 \Rightarrow x=1 \pm \sqrt{1-2}=1 \pm i \\
\end{align*}

\textbf{Exempel, Heltalslösning med okänd konstant:}
\begin{align*}
  &x^3 + bx^2 - 7x -7 = 0 \text{ har en heltals lösning} \\
  &\text{Eftersom ekvationen har en heltals lösning} \\
  &\text{ vet vi att } \alpha \mid 7\\
  &\text{Kandidaterna är } \{1,-1,7,-7 \} \\
  &I   (x= 1) \;  1 + b\cdot  1 - 7 -7 = 0 \Rightarrow b=13 \in \mathbb{Z} \\
  &II  (x=-1) \; -1 + b\cdot -1 + 7 -7 = 0 \Rightarrow b= 1 \in \mathbb{Z} \\
  &III (x= 7) \;  7 + b\cdot  7 -49 -7 = 0 \Rightarrow b=\_ \notin \mathbb{Z} \\
  &IV  (x=-7) \;  7 + b\cdot -7 +49 -7 = 0 \Rightarrow b=\_ \notin \mathbb{Z} \\
  &\text{svar: b=13, b=1}
\end{align*}

\textbf{Exempel, Rationell root:}
\begin{align*}
  &g(x)= 2x^3 +2x^2 +2x +3 = 0 \text{ har en rationell root} \\
  &\text{Eftersom ploynomet har en rattionel root vet vi att: } \\
  &\frac{p}{q} \; p,q \in \mathbb{Z}, \; SGD(p,q) = 1, \; p \mid 3 \land q \mid 2 \; (a_0) \\
  &\text{Det möjliga delarna är } p=\pm 1,\pm 3, \; q=\pm 1,\pm 2 \\
  &x=-\frac{3}{2} \text{ är den enda av kandidaterna som ger en} \\
  &\text{sann root} \\
  &\text{Vilket vi löser genom polynom divition}
\end{align*}


\textbf{Exempel, Renimaginär root:}
\begin{align*}
  &z^4 + 4z^3 + 8z^2 + 12z +15 = 0 \text{ har en renimachinär root} \\
  &\quad  \text{Eftersom ploynomet har en rentimaghinär root vet vi att } \\
  &z=bi, b \in \mathbb{R}, b \neq 0 \\
  &0= b^4i^4 +4b^3i^3 +8b^2i^2 +12bi +15 \\
  &= b^4 -4b^3i -8b^2 +12bi +15 \\
  &= (b^4-8b^2+15) + (12b-4b^3)i \\
  &=0= b^4-8b^2+15 \\
  &0= 12b-4b^3 = 4b(3-b^2) \Rightarrow b=0, b=\sqrt{3}, b=-\sqrt{3} \\
  &b=0 \text{ är ej en giltig lösning.} \\
  &\text{Enlight faktorsatsen får vi följande} \\
  &p(z)=(z-\sqrt{3}i)(z+\sqrt{3}i)Q(x) = (z^2+3)Q(z) \\
  &\text{Vilket vi löser genom polynom divition}
\end{align*}

\textbf{Exempel, Dubel root:}
\begin{align*}
  &p(t)= t^3 -5t^2 +3t +9 = 0 \text{ har en dubbel root} \\
  &\text{Eftersom ploynomet har en dubbel root vet vi att} \\
  &\text{derivatan ger os en root} p'(t)=0  \\
  &\text{Att räkna ut derivatan gör man genm formeln } \\
  &ax^b \Rightarrow b\cdot ax^{b-1} \\
  &p'(t) = 3t^2 -10t +3 = 0 \Rightarrow t_1=3,t_2=\frac{1}{3} \\
  &\text{ dock är sista en falsk root} \\
  &p(t) = {x-3}^2Q(t) \\ 
  &\text{Vilket vi löser genom polynom divition}
\end{align*}

\textbf{Exempel, SGD polynom:}
\begin{align*}
  &\text{Förenkla } \frac{x^4+x^3+2x-4}{x^4-x^3-2x-4} \\
  &SGD(x^4+x^3+2x-4, x^4-x^3-2x-4):  \\
  &x^4+x^3+2x-4 = 1 \cdot x^4-x^3-2x-4 + (2x^3 +4x) \\
  &x^4-x^3-2x-4 = \frac{x}{2}-\frac{1}{2} \cdot (2x^3 +4x) + (-2x^2-4) \\
  &\text{ med polynom divition} \\
  &2x^3 +4x = -x \cdot (-2x^2-4) +0 \\
  &\text{Med SGD så kan vi ta ett polynom som är hjämt delbart} \\
  &\text{ex:} -2(x^2+x) \\
  &\text{polynom divition: } \frac{x^4+x^3+2x-4}{x^2+x} = x^2+x-2 \\
  &\text{polynom divition: } \frac{x^4-x^3-2x-4}{x^2+x} = x^2-x-2 \\
  &\frac{x^4+x^3+2x-4}{x^4-x^3-2x-4} = \frac{x^2+x-2}{x^2-x-2} \\  
\end{align*}
\end{multicols}
\raggedcolumns

