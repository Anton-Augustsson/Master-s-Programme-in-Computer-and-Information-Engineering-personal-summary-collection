\chapter{Imperativ and Object-Oriented Programming Methodology}

\newpage

% Resorces
% http://wrigstad.com/ioopm18/course-report2018.html
% http://wrigstad.com/ioopm19/
% TODO: write more relevant data like pointer arithmetics more about data structure and java differs from c 

\begin{multicols}{2}
\section{Introduction}

\subsection{Test}
Innan: 
Läs igenom standard instruktionerna.
Sätt upp mjukvaru miljö på skollans datorer.

Under:
Det är mycket tid. 
Läs igenom instruktionerna på uppgiften nogrant.



\section{Imperative programming i C}
C is a strict programming language, we must therefore write every type of variables.
Like any other imperative programming language it execute sequentially and every argument is evaluated
in order. It also allow side effects like changing the state freely. 

\noindent\textbf{Vilkorsatser}
\begin{lstlisting}[language=C]
if (expr) { expr; } 
if (age > 100) { puts("Very old"); } 

if (expr) { expr; } 
else { expr; }
if (age % 2 == 0) { puts("Even"); } 
else { puts("Odd"); }

expr ? expr : expr
a < b ? b : a;
\end{lstlisting}

\noindent\textbf{Shortcuts loop operations}
\begin{lstlisting}[language=C]
n_fakultet *= n--;
\end{lstlisting}


\subsection{Data structurer}
\subsubsection{stack och heap}
I kortids minnet som c kompilatorn hanterar skjälv så sappar datan i en stack.
Stack funkar som talrikar som man staplar på varandra och tar den som du la senast ut.

Heap är för långtids mine och störe data som arrayer.
Det fungerar som ett rutat papper där varge ruta är en viss data size.

\noindent\textbf{Shortcuts loop operations}
För att andvända heapen, (rita på det rutade papret) måste man följa 4 steg.
\noindent\emph{ steg 1: räkna ut hur många rutor vi behöver} \newline
Använd plattformsoberoende hjälpmedel och vanlig aritmetik
\begin{lstlisting}[language=C]
  sizeof(T) * antal
  int size = sizeof(int) * 1024
\end{lstlisting}

\noindent\emph{ steg 2: reservera motsvarande yta} \newline
Här kan det gå fel — det kanske inte finns plats på papperet
\begin{lstlisting}[language=C]
  T *namn = malloc(antal bytes); 
  int *skonummer = malloc(size);
\end{lstlisting}

\noindent\emph{ steg 3: använd ytan hejvilt} \newline
Men sudda först!
(Beror på datastrukturen)

\noindent\emph{ steg 4: lämna tillbaka platsen när du är klar} \newline
Annars kommer det gå dåligt i ett framtida steg 2
\begin{lstlisting}[language=C]
  free(namn); 
  free(skonummer);
\end{lstlisting}


\subsubsection{Structar}
För att skapa egna data structurer som andvänder man sig av ``struct'' operatorn.
Om man vill gömma fler data typer i en så kan man andvända en ``union''.
För att göra en typ så andväder man ``typedef'' och avslutar med \begin{verbatim}namnet_{t}\end{verbatim}

\subsubsection{Pekare}
Pekare pekar var datan finns så man kan skicka information utan att sicka stora mängder data.

\begin{lstlisting}[language=C]
  int  a  
  int *b  
\end{lstlisting}
%// a är en variabel som innehåller ett heltal
%// b är en variabel som innehåller en adress till en plats i minnet där det finns ett heltal
Syntaxen som andvänds är:
pekartypen \begin{verbatim} int * \end{verbatim}, adresstagningsoperatorn \begin{verbatim} & \end{verbatim} och avrefereringsoperatorn \begin{verbatim} * \end{verbatim}.
  Skillnaden mellan arrayer och pointers är väldigt liten i c.
  Därför är samma oporation för string (\begin{verbatim}char *\end{verbatim})
som en pointer (\begin{verbatim} int * \end{verbatim}). Pekaren behöver inte peka på en data värde utan kan peka på flertal värden.


%https://en.wikipedia.org/wiki/C_data_types

\subsubsection{Linked list}
%https://www.youtube.com/watch?v=VOpjAHCee7c
Varge element har data av någon typ samt en pekare till nästa element.
Head and tail kan beskriva en sådan lista, då head är första ellementet
och tail är rästen.

Tid komplexiteten är: O\(n\). Linjär


\subsubsection{Hash table}
%https://www.youtube.com/watch?v=2Ti5yvumFTU&list=WL&index=5&t=686s
Hash table är ett mer effectivt sätt att hantera data istället för en array.
Då man inte behöver kålla varge element vad den finns utan andvänder en hash
function för att skapa ett ``nummer'' som pekar till vad datan finns någonstans
exempelvis kan det vara mod element i hash table då vet vi har vi ska ska söka efter.
Ett problem som kommer upstå är att data mappas till samma plats, då får man andvända
exempelvis \emph{linuer probing} som säger att om det är fult på platsen gå till nästa tills
det finns och om det inte finns så protesterar programet. Varge element inehåller inget värde
eller deleted varde om det är tomma.

Tid komplexiteten är: O\(1\). Konstant men inte i verkligheten


% F12

\section{Object-Oriented programming i Java}

%arv, amerikanska skollan ärver kåd, is skandinaviska skollan ärver betende.
%mäniska ärver från däggjur och däggjur ärver från djur, liv, ting....

%i java fiins supper klasser som man inte kan skapa en nytt dådant objekt men du andvändar classarna som extendare som du skapar.

% konstruktor är den metod som initialiserar objectet
% supper
% finnal value wont change like constant, class ej ervs
% abstarct cant creat new objects (ingen instans)

% man kan inte ha sama metod namn och argument men olika retur typer blir overload
% Tillåtet i vissa språ
kör den mest spesifica implementationen av metoden
single dispage, vi bryr os ombjectet vi kör metoden på men inte argumentet
\end{multicols}
\raggedcolumns