\chapter{Computer System with Project Work}

\newpage

\section{Conclusion}
User mode and kernal mode, how programs interacts with the kernal mode. What happens
beneth the hode how the kurnal organise processes use ques for IO operation on nowing
what processes to prioritice. How routing works. Treads is part of a process but one can
manage them in user mode (mostly). But syncronasition bust be maintain or the data may not
be correct. How the memmory works. The security problems Alice and Bob, encryptino.
\subsection{Man-in-middle attack}
the attacker secretly relays and possibly alters the communications between two parties who believe that they are directly communicating with each other
\subsection{Chosen-plaintext attack}
presumes that the attacker can obtain the ciphertexts for arbitrary plaintexts. The goal of the attack is to gain information that reduces the security of the encryption scheme
\subsection{Side-channel attack}
any attack based on information gained from the implementation of a computer system, rather than weaknesses in the implemented algorithm itself
Timing information, power consumption, electromagnetic leaks or even sound can provide an extra source of information, which can be exploited.
\subsection{Replay attack}
valid data transmission is maliciously or fraudulently repeated or delayed. This is carried out either by the originator or by an adversary who intercepts the data and re-transmits it, possibly as part of a spoofing attack by IP packet substitution. This is one of the lower-tier versions of a man-in-the-middle attack. Replay attacks are usually passive in nature.
